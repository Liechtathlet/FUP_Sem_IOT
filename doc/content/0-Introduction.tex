% !TeX encoding=utf8
% !TeX spellcheck = de_CH_frami

\chapter{Einleitung}
Diese Arbeit wurde als Seminararbeit zur Vorlesung von funktionalen Programmiersprachen verfasst.
In diesem Kapitel wird die Aufgabenstellungen und Rahmenbedingungen der Arbeit erläutert.


\section{Hintergrund}
Einer immer grösseren Beliebtheit erfreuen sich kleine Alltagsgegenstände welche mit dem Internet verbunden sind. Dieser Bereich wird \Gls{gls:iot} genannt. Diese Gegenstände sind in der Lage Daten zu erheben und weiterzuleiten. Da es zukünftig voraussichtlich immer mehr \Gls{gls:iot} Gegenstände geben wird fallen immer mehr Daten an. 
Diese Daten werden wegen ihrer Masse auch BigData genannt.

In dieser Arbeit wird evaluiert wie sich funktionale Programmiersprachen im Bezug auf \Gls{gls:iot} eignen, um BigData auszuwerten.


\section{Ziel}
Mit einem \Gls{gls:iot} Gerät sollen Daten aufgezeichnet werden. Diese werden als BigData gesammelt und sollen mit einer funktionalen Programmiersprache ausgewertet und ansprechend ausgegeben werden.

Das Hauptziel der Arbeit besteht darin zu überprüfen wie geeignet funktionale Programmiersprachen für die Auswertung von BigData sind. Als Nebenziel soll evaluiert werden, ob eine funktionale Programmiersprache zum erfassen von Daten auf einem \Gls{gls:iot} Gerät verwendet werden kann.

\section{Aufgabenstellung} \label{sec:Aufgabenstellung}
Die freigegebene Aufgabenstellung lautet wie folgt:

\begin{itemize}
\item Projektname: Seminar BigData mit RaspberryPi und F\#
\item Ausgangslage: Durch die rasante Entwicklung im Bereich \Gls{gls:iot} ergeben sich viele neue Anwendungsmöglichkeiten. Da der RaspberryPI immer leistungsfähiger geworden ist, soll evaluiert werden ob er sich für den Einsatz von funktionalen Sprachen im Bereich BigData eignet.
\item Ziel der Arbeit: Es soll gezeigt werden wie F\# Sharp auf einem RaspberryPi im Bereich BigData und \Gls{gls:iot} eingesetzt werden kann.
\item Aufgabenstellung: Es soll gezeigt werden, wie eine funktionale Programmiersprache (F\#) im Kontext von BigData und \Gls{gls:iot} eingesetzt und verwendet werden kann. Es soll eine Anwendung zur Sammlung von Sensordaten auf einem Raspberry PI und eine Anwendung zur Analyse / Auswertung der gesammelten Daten implementiert werden.
\end{itemize}


\section{Erwartete Resultate} \label{sec:ErwarteteResultate}
Gemäss freigegebener Aufgabenstellung werden folgende Resultate erwartet:

\begin{itemize}
\item Dokumentation
\item Implementation / Prototyp
\end{itemize}


\section{Abgrenzung} \label{sec:Abgrenzung}
Aufgrund des Umfanges der Arbeit und der begrenzten Zeitdauer werden folgende Punkte von der Arbeit abgegrenzt:

\begin{itemize}
\itemBfText{Schnittstellendokumentation}{In dieser Arbeit werden nicht die Schnittstellendokumentationen und -spezifikationen rekonstruiert. Es werden jeweils die relevanten Aspekte betrachtet und hervorgehoben.}

\end{itemize}


\section{Motivation}


\section{Struktur}


\section{Planung} \label{sec:Intro:Planning}
