% !TeX encoding=utf8
% !TeX spellcheck = de_CH_frami

\chapter{Einleitung}
Diese Arbeit wurde als Seminararbeit zur Vorlesung "`Funktionale Programmierung"' verfasst.
In diesem Kapitel werden die Aufgabenstellung und die Rahmenbedingungen der Arbeit erläutert.


\section{Hintergrund}
Einer immer grösseren Beliebtheit erfreuen sich kleine Alltagsgegenstände welche mit dem Internet verbunden sind. Dieser Bereich wird \gls{acr:IOT} genannt. Diese Gegenstände / Geräte sind in der Lage Daten zu sammeln, speichern und weiterzuleiten. Da es zukünftig voraussichtlich immer mehr solcher \gls{acr:IOT} Geräte geben wird fallen immer mehr Daten an. Der Themenbereich des Aufbereiten, Aggregieren, Analysieren und Visualisierens dieser Daten wird auch als "`BigData"' bezeichnet.


\section{Ziel}
Das Hauptziel dieser Arbeit ist die Überprüfung des Sachverhaltes, ob sich funktionale Programmiersprachen eigenen um Sensor-Daten auf einem \gls{acr:IOT} Gerät zu sammeln und anschliessend auszuwerten. Um diesen Sachverhalt zu prüfen, sollen mit einem \gls{acr:IOT} Gerät (Raspberry Pi) Sensor-Daten ermittelt und gespeichert werden. Anschliessend folgt eine Aufbereitung und Visualisierung der gespeicherten Daten.

\section{Aufgabenstellung} \label{sec:Aufgabenstellung}
Die freigegebene Aufgabenstellung lautet wie folgt:

\begin{itemize}
\item Projektname: Seminar BigData mit RaspberryPi und F\#
\item Ausgangslage: Durch die rasante Entwicklung im Bereich \Gls{gls:iot} ergeben sich viele neue Anwendungsmöglichkeiten. Da der Raspberry Pi immer leistungsfähiger geworden ist, soll evaluiert werden ob er sich für den Einsatz von funktionalen Sprachen im Bereich BigData eignet.
\item Ziel der Arbeit: Es soll gezeigt werden wie F\# Sharp auf einem Raspberry Pi im Bereich BigData und \gls{acr:IOT} eingesetzt werden kann.
\item Aufgabenstellung: Es soll gezeigt werden, wie eine funktionale Programmiersprache (F\#) im Kontext von BigData und \gls{acr:IOT} eingesetzt und verwendet werden kann. Es soll eine Anwendung zur Sammlung von Sensordaten auf einem Raspberry Pi und eine Anwendung zur Analyse / Auswertung der gesammelten Daten implementiert werden.
\end{itemize}


\section{Erwartete Resultate} \label{sec:ErwarteteResultate}
Gemäss freigegebener Aufgabenstellung werden folgende Resultate erwartet:

\begin{itemize}
\item Dokumentation
\item Implementation / Prototyp
\end{itemize}


