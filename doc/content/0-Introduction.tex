% !TeX encoding=utf8
% !TeX spellcheck = de_CH_frami

\chapter{Einleitung}
Diese Arbeit wurde als Seminararbeit zur Vorlesung von funktionalen Programmiersprachen verfasst.
In diesem Kapitel wird die Aufgabenstellungen und Rahmenbedingungen der Arbeit erläutert.

\section{Hintergrund}
Einer immer grösseren Beliebtheit erfreuen sich kleine Alltagsgegenstände welche mit dem Internet verbunden sind. Dieser Bereich wird \Gls{gls:iot} genannt. Diese Gegenstände sind in der Lage Daten zu erheben und weiterzuleiten. Da es zukünftig voraussichtlich immer mehr \Gls{gls:iot} Gegenstände geben wird fallen immer mehr Daten an. 
Diese Daten werden wegen ihrer Masse auch Big Data genannt.

In dieser Arbeit soll evaluiert werden wie sich funktionale Programmiersprachen im Bezug auf \Gls{gls:iot} eignen, um Big Data auszuwerten.


\section{Ziel}
Mit einem \Gls{gls:iot} Gerät sollen Daten aufgezeichnet werden. Diese werden als Big Data gesammelt und sollen mit einer funktionalen Programmiersprache ausgewertet und ansprechend ausgegeben werden.

Das Hauptziel der Arbeit besteht darin zu überprüfen wie geeignet funktionale Programmiersprachen für die Auswertung von Big Data sind. Als Nebenziel soll evaluiert werden, ob eine funktionale Programmiersprache zum erfassen von Daten auf einem \Gls{gls:iot} Gerät verwendet werden kann.

\section{Aufgabenstellung} \label{sec:Aufgabenstellung}
Die freigegebene Aufgabenstellung lautet wie folgt:

\begin{itemize}
\item 
\end{itemize}


\section{Erwartete Resultate} \label{sec:ErwarteteResultate}
Gemäss freigegebener Aufgabenstellung werden folgende Resultate erwartet:

\begin{itemize}
\item 
\end{itemize}


\section{Abgrenzung} \label{sec:Abgrenzung}
Aufgrund des Umfanges der Arbeit und der begrenzten Zeitdauer werden folgende Punkte von der Arbeit abgegrenzt:

\begin{itemize}
\itemBfText{Schnittstellendokumentation}{In dieser Arbeit werden nicht die Schnittstellendokumentationen und -spezifikationen rekonstruiert. Es werden jeweils die relevanten Aspekte betrachtet und hervorgehoben.}

\end{itemize}


\section{Motivation}

\section{Struktur}


\section{Planung} \label{sec:Intro:Planning}
