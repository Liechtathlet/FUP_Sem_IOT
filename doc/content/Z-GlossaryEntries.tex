% !TeX encoding=utf8
% !TeX spellcheck = de_CH_frami

%LOOK: http://tex.stackexchange.com/questions/8946/how-to-combine-acronym-and-glossary

%%% --- Glossary entries


\newglossaryentry{gls:iot}{name=IoT,
description={Unter Internet of Things (IoT) versteht man Alltagsgegenstände, welche mit dem Internet verbunden sind. Im Gegensatz zu normalen Computern, mit denen man sich aktiv beschäftigt, soll IoT den Menschen unterstützen in seinem Tun, ohne dass dieser etwas davon merkt.}}

\newglossaryentry{gls:ERP}{name=ERP,
description={Ein klassiches \gls{acr:ERP} System dient dazu interne Prozesse in den Kernbereichen Finanz- und Rechnungswesen, Vertrieb \& Marketing, Personalwesen und Produktion zentral in einem System abzubilden. Heutige \gls{acr:ERP} Systeme bieten neben den Kernbereichen noch zahlreiche weitere Funktionen, welche die klassischen Geschäftsprozesse unterstützen. \cite[S. 486]{Pearson:WirtschaftsInf:LaudonEtAl}}}


%%% --- Acronym definitions
\IfDefined{newacronym}{

\newacronym{acr:ZHAW}{ZHAW}{Zürcher Hochschule für Angewandte Wissenschaften}

\newacronym{acr:EBS}{EBS}{Einschreibe- und Bewertungssystem}

\newacronym{acr:I2C}{I2c}{Inter-Integrated Circuit}

\newacronym{acr:CSV}{CSV}{Comma Separated Values}

\newacronym{acr:XML}{XML}{Extensible Markup Language}

\newacronym{acr:JSON}{JSON}{JavaScript Object Notation}

\newglossaryentry{acr:ERP}{type=\acronymtype, name={ERP}, description={Enterprise-Resource-Planning}, first={Enterprise-Resource-Planning (ERP) \glsadd{gls:ERP}},see=[Glossary:]{\gls{gls:ERP}}}

}





