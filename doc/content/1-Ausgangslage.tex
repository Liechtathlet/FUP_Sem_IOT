% !TeX encoding=utf8
% !TeX spellcheck = de_CH_frami

\chapter{Recherche}
In diesem Kapitel werden die Grundlagen recherchiert wie die beiden Teilprojekte "`Sensordaten sammeln"' und "`Sensordaten auswerten (BigData)"' angegangen werden könnten.

\section{Ausgangslage}
Die Vorlesung zu diesem Semniar befasst sich mit den Konzepten der Funktionalen Programmierung. Zur Veranschaulichung dieser Konzepte wurde die Programmiersprache F\# des .NET-Frameworks verwendet. Aufgrund dessen haben wir uns entschieden auch dieses Seminar mit der uns nun bekannten Sprache F\# umzusetzen. Als \Gls{gls:iot} Gerät wird ein Raspberry Pi\footcite{Raspberry_Pi_2016-04-24} verwendet. Der Raspberry Pi ist ein Einplatinencomputer welcher von der britischen Raspberry Pi Foundation entwickelt wurde. Der Raspberry Pi bietet den Vorteil, dass er sehr weit verbreitet ist\footcite{Raspberry_Pi_Erfolgsgeschichte_2016-04-24}, er kostengünstig ist und es inzwischen eine sehr grosse Anzahl an Sensoren auf dem Markt gibt mit welchem man Daten sammeln kann\footcite{Raspberry_Pi_Sensor_2016-04-24}.

\section{Der Raspberry Pi}
\label{sec:recherche:rpi}
Wie bereits im vorangehenden Kapitel beschrieben, handelt es sich beim Raspberry Pi um einen Einplatinencomputer. Dieser Einplatinencomputer bietet verschiedene zentrale Hardware-Schnittstellen um externe Geräte für Input und Output anzuschliessen.

Vom Raspberry PI gibt es folgende Modelle:
\begin{itemize}
\item Raspberry Pi Compute Module
\item Raspberry Pi Zero
\item Raspberry Pi Model A
\item Raspberry Pi Model A+
\item Raspberry Pi Model B
\item Raspberry Pi Model B+
\item Raspberry Pi 2 Model B
\item Raspberry Pi 3 Model B
\end{itemize}

\todo{Evtl. ein Bild / List mit Daten dazu?}

Am 29 Februar 2016 ist die neuste Version, der Raspberry Pi 3 (Model B), auf dem Markt erschienen\footcite{Raspberry_Pi_3_2016-04-24}. Einige Zahlen zu dem Gerät:

\begin{itemize}
\item 1.2GHz 64-bit quad-core ARM Cortex-A53 CPU (~10x die Leistung eines Raspberry Pi 1 und 50-60\% die Leistung eines Raspberry Pi 2)
\item Integriertes 802.11n wireless LAN und Bluetooth 4.1
\item Komplette Kompatibilität zu Raspberry Pi 1 und 2 (Model B)
\end{itemize} 


\section{F\# mit dem Raspberry Pi}
\label{sec:recherche:fsharprpi}
Um F\# auf dem dem Raspberry PI auszuführen, gibt es grundsätzlich zwei Möglichkeiten, welche nachfolgend erläutert werden.
Da es sich bei F\# um eine Sprache des Microsoft .NET-Frameworks handelt, wird für die Ausführung zwingend eine Implementierung des .NET-Frameworks benötigt.

\subsection{Linux (Raspbian)}
Ein weit verbreitetes Betriebsystem für den Raspberry Pi ist das Raspbian\footcite{FrontPage_-_Raspbian_2016-04-24} OS. Bei dem Namen handelt es sich um eine Zusammenfassung von Raspberry und Debian. Demnach handelt es sich auch um eine Debian Distribution, welche spezifisch für den Raspberry Pi entwickelt wurde.

\subsubsection{Mono-Framework}
Eine Möglichkeit um unter Linux, beziehungsweise Raspbian, F\# auszuführen ist das Mono-Framework\footcite{Mono_2016-04-24}. Dabei handelt es sich um eine Open Source Implementierung von Microsoft's .NET Framework.

\subsubsection{Mono-Framework + .NET Core}
Eine weitere Möglichkeit besteht aus einer Kombination des aktuellen Mono-Frameworks und der aktuellen Open Source Implementierung für Linux von .NET Core von Microsoft.


\subsection{Windows 10 IoT}
Microsoft hat mit Windows 10 IoT eine Version ihres Betriebssystem herausgebracht, welches speziell für leistungsschwächere Geräte entwickelt wurde\footcite{Windows_IoT_2016-04-24}. Bei der IoT Version von Windows 10 ist das .NET Framework bereits standardmässig an Bord. Demnach sollte es keine Probleme geben um F\# auf dieser Plattform zu betreiben.







\todo{Dokumentieren wir den SW-Setup der Hardware? Installationsanleitung?}

