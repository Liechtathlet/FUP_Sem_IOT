% !TeX encoding=utf8
% !TeX spellcheck = de_CH_frami

\chapter{Datensammlung}
Dieses Kapitel beschäftigt sich mit dem Teilprojekt der Datensammlung. 
Dazu wird ein Raspberry Pi verwendet welcher mit F\# Informationen von verschiedenen Sensoren zusammentragen soll.
Diese Daten sollen in darauf folgenden Kapiteln weiterverwendet werden.

\section{Verwendete Hardware}
Für diese Seminararbeit wurden folgende Hardwarekomponenten verwendet:

\begin{itemize}
\item Raspberry Pi 2 Model B
\item Raspberry Pi 3 Model B
\item GrovePi Sensoren
\end{itemize}


\subsection{GrovePi Sensoren}
Für den Raspberry Pi gibt es viele Sensoren auf dem Markt. Herauskristallisiert hat sich jedoch das Starter Kid GrovePi+\footcite{GrovePi_2016-04-24}. Dieses beinhaltet unter anderem Ton-, Temperatur-, Feuchtigkeit- und Lichtsensoren. Der Vorteil an den GrovePi Sensoren besteht an den geringen Kosten, dem einfachen Anschluss an den Raspberry Pi und die vielen verfügbaren Beispiele in unterschiedlichsten Programmiersprachen.

Die Recherchen haben ebenfalls gezeigt, dass es eine Library für .NET gibt mit welchem die Sensoren angesprochen werden können\footcite{NuGet_GrovePi_2016-04-24}.


\section{Verwendete Software}
\subsection{Datenauswertung}
Die von den Sensoren gespeicherten Daten werden in einem noch zu definierenden Format abgespeichert und danach für die Datenauswertung ausgelesen. Dafür wurde vom Dozenten die Library F\# Data vorgeschlagen\footcite{Fsharp_Data_2016-04-24}. Diese kann Daten im Format CSV, HTML, JSON und XML entgegennehmen und für die Verwendung in F\# zur Verfügung stellen.

\section{F\# auf dem Raspberry Pi}
Im \cref{sec:recherche:fsharprpi} \nameref{sec:recherche:fsharprpi} wurden zwei Möglichkeiten aufgezeigt, welche es ermöglichen F\# auf einem Raspberry Pi laufen zu lassen.
Entweder wird das Linux Raspbian mit Mono verwendet, oder das Windows 10 IoT.

Es war zu erwarten, dass der Weg über Window 10 IoT der einfachere ist, da dort .NET schon mitgeliefert wird. 
In diesem Projekt wurden beide Methoden ausprobiert. 
Die folgenden Abschnitte erläutern mit den beiden Herangehensweisen an das Problem.

\subsection{Raspbian}

\subsection{Windows 10 IoT}
Windows 10 IoT ist eine Version des Betriebssystems von Microsoft, welches speziell für kleinere Geräte mit weniger Rechenleistung konzipiert wurde.

Die Installation gemäss der Anleitung auf dem Github Account from Microsoft\footcite{install_win10iot_2016-04-25} war nicht erfolgreich. 
Der Raspberry Pi startete nicht und blieb beim Rainbow Screen\footcite{RPi_Rainbowscreen_2016-04-25} hängen. 
Mit dem NOOBS\footcite{NOOBS_2016-04-25} Installer, welcher von der Raspberry Pi Foundation zur Verfügung gestellt wird, war die installation von Windows 10 Io