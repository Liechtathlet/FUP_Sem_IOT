% !TeX encoding=utf8
% !TeX spellcheck = de_CH_frami

\chapter{Sensordaten sammeln}
Dieses Kapitel beschäftigt sich mit dem Teilprojekt der Sammlung von Sensordaten. Die Aufbereitung der gesammelten Daten ist Teil eines weiteren Kapitels.


\section{Analyse \& Setup}
\section{F\# auf dem Raspberry Pi}
Im \cref{sec:recherche:fsharprpi} \nameref{sec:recherche:fsharprpi} wurden zwei Möglichkeiten aufgezeigt, welche es ermöglichen F\# auf einem Raspberry Pi laufen zu lassen.

Es ist zu erwarten, dass der Weg über Window 10 IoT der einfachere sein wird, da dort das benötigte .NET Framework Teil des Systems ist. In dieser Seminararbeit wurden bewusst beide Varianten (Linux (Raspbian) und Windows 10 IoT) ausprobiert und getestet. Die folgenden Abschnitte erläutern den Setup und die Erkenntnisse aus dem Test der beiden Varianten. 

\subsection{Linux (Raspbian Jessie)}
Die Installation und der Betrieb des Mono-Frameworks unter Linux, beziehungsweise Raspbian Jessie, ist ohne grossen Aufwand möglich. Die Installation erfolgt über die bereitgestellten Packages aus dem Software-Repository.

\todo{Machen wir hier eine detailliertere Setup Anleitung?}
\subsection{1: Installation \& Konfiguration von Raspbian Jessie}
Zuerst wurde Raspbian Jessie auf dem Raspberry Pi installiert. Für die Installation des Betriebssystems wurde NOOBS (New Out Of Box Software) verwendet. NOOBS ist ein Installationsmanager für Betriebssysteme, der es ermöglicht per Klick ein gewünschtes Betriebssystem zu installieren.

% NOOBS: https://www.raspberrypi.org/documentation/installation/noobs.md

% WLAN-Problem (IPv6)
\subsection{2: Installation des Mono-Frameworks \& F\#}
Im Anschluss wurde die neuste Version des Mono-Frameworks (4.2.3) und der F\#-Interactive Shell (F\# Version 4.0) über die in den Software-Repositories vorhandenen Pakete installiert.

\subsection{3: Installation \& Konfiguration GrovePi}
Die Installation und Konfiguration wurde anhand der vom Hersteller zur Verfügung gestellten Anleitung durchgeführt. Der Setup wurde anschliessend mit einem bereitgestellten Python-Script getestet.
% Install Instructions: http://www.dexterindustries.com/GrovePi/get-started-with-the-grovepi/setting-software/

% Probleme: Clone Repo, git repo nicht von überall erreichbar (Timeout)

\subsection{4: Anbindung des GrovePi via GrovePi-NuGet-Library}
Nun galt es den GrovePi über F\# anzusteuern und entsprechende Sensordaten auszulesen. Während der Recherchen wurden wir auf eine NuGet-Library\footcite{NuGet_GrovePi_2016-04-24} aufmerksam, welche die entsprechende Funktionalität bereitstellt. In der Entwicklungsumgebung (Monodevelop) wurde eine neues F\# Projekt angelegt und das entsprechende NuGet-Packet installiert. Der erste testweise Build mit dem NuGet-Paket schlug fehl. Gemäss der Fehlermeldung wird für dieses Paket das Microsoft .NET Framework 5 benötigt. Beim Microsoft .NET Framework handelt es sich um eine Neuauflage des .NET Frameworks.

\subsubsection{.NET Core}
\todo{Kurze Beschreibung / Hintergrund zu .NET}
% https://de.wikipedia.org/wiki/.NET_Framework

.NET Core ist der nächste grosse Evolutionsschritt des .NET Frameworks.

-Open Source, .NET Framework Implementation, Plattformneutral, Portabel, im Gegensatz zum klassischen .NET Framework: modular
-.NET core apps: run on 32-bit and 64-bit


%https://de.wikipedia.org/wiki/.NET_Core
%https://www.dotnetfoundation.org/netcore
% https://dotnet.github.io/
% https://dotnet.github.io/docs/getting-started/what-is-dotnet.html
% https://blogs.msdn.microsoft.com/bethmassi/2015/02/25/understanding-net-2015/

\subsection{5: Installation von .NET Core}
Aufgrund der Fehlermeldung haben wir nun versucht .NET Core auf dem Raspberry zu installieren.


Problem: Version Manager, Build-System
dnx, dnvm
%  https://blogs.msdn.microsoft.com/bethmassi/2015/02/25/understanding-net-2015/
 % - http://docs.asp.net/en/latest/getting-started/installing-on-linux.html
  %- http://docs.asp.net/en/latest/dnx/projects.html
  %- https://blogs.msdn.microsoft.com/bethmassi/2015/02/25/understanding-net-2015/
  %- https://github.com/robsonj/GrovePi
  %- http://oren.codes/2015/07/29/targeting-net-core/
  %- http://www.paraesthesia.com/archive/2015/10/20/gotcha-with-dnx-dependency-resolution-dotnet-pcl/
  %- http://davidfowl.com/diagnosing-dependency-issues-with-asp-net-5/
  %- https://github.com/mrward/monodevelop-dnx-addin
  %- http://dotnet.github.io/docs/getting-started/installing/installing-core-linux.html
  %- https://www.dotnet.xyz/tutorials/net-core-unter-linux-raspberry-pi/
  %- https://www.hackster.io/9381/grove-pi-windows-iot-getting-started-94bf38
  
    - .NET Core aktuell nur für 64bit
    - Raspberry PI OS sind praktisch alle 32bit
    - GrovePi NuGet Library verwendet .NET Core native (Windows only)
    - https://github.com/raspberry-sharp/raspberry-sharp-io
\subsection{6: Finaler Setup}



\subsection{Windows 10 IoT}
Windows 10 IoT ist eine Version des Betriebssystems von Microsoft, welches speziell für kleinere Geräte mit weniger Rechenleistung konzipiert wurde.

Die Installation gemäss der Anleitung auf dem Github Account from Microsoft\footcite{install_win10iot_2016-04-25} war nicht erfolgreich. 
Der Raspberry Pi startete nicht und blieb beim Rainbow Screen\footcite{RPi_Rainbowscreen_2016-04-25} hängen. 
Mit dem NOOBS\footcite{NOOBS_2016-04-25} Installer, welcher von der Raspberry Pi Foundation zur Verfügung gestellt wird, war die installation von Windows 10 Io


\section{Umsetzung}

\subsection{Verwendete Software}
Für die Realisierung werden folgende Softwarekomponenten verwendet:

\begin{itemize}
\item Raspbian Jessie
\item GrovePi Firmware + Beispiele
\item Mono 4.2
\end{itemize}





